
{\actuality} Фазовые переходы в системах конечного числа частиц чрезвычайно широко распространены в естественной природе, и не представляет труда найти множество примеров таких превращений. Например, стоит вспомнить таяние крохотной снежинки и превращение ее в капельку воды, для фазового перехода первого рода в системе конечного числа атомов, или же спонтанное возникновение магнитного момента у однодоменной наночастицы, в случае фазового перехода второго рода. 

Наличие теоретической или экспериментальной фазовой диаграммы для конкретной модели обычно означает завершенность теории и глубокий уровень нашего понимания природы явлений в системах многих взаимодействующих тел. Одной из самых простых моделей взаимодействующих многих тел является модель Изинга. Оказалось, не смотря на более чем вековую историю исследования этой самой простой модели c взаимодействием ближайших соседей из первой координационной сферы, даже в самом простом случае, в отсутствии внешнего магнитного поля, оказывается не так просто с помощью таких мощных и зарекомендовавших себя в теоретической физике методов, построить теоретическую магнитную фазовую диаграмму, например, в осях ''Температура---Вероятность'', $T(P_+)$. Под температурой $T$ мы понимаем критическую температуру фазового превращения, под вероятностью $P_+$ понимаем вероятность того, что случайно взятая пара взаимодействующих спинов имеет ферромагнитную связь $+J$, т.е. значение обменной константы $J>0$. Очевидно, что для ферромагнетика $P_+=1$, для антиферромагнетика $P_+=0$, для спинового стекла $P_+=0.5$.

Подходы к завершению теории, т.е. к построению фазовой диаграммы продолжаются и с помощью аналитических методов \cite{belokon2006}, численными Монте-Карло расчетами \cite{Hasenbusch2008}, и экспериментально \cite{Mirebeau2022}. Как будет показано ниже прямые строгие вычисления статистической суммы в модели Изинга для конечного числа взаимодействующих частиц позволяют решить множество задач без привлечения методов теории нарушения симметрии реплик (RSB) \cite{newman2023proof}.

В данной работе мы вычислили границы между ферромагнитным, парамагнитным, спинстекольным и антиферромагнитным состояниями. Эдвардс и Андерсон показали, что в таких магнетиках возможен новый тип фазового перехода, который связан с замораживанием спинов \cite{edwards1975theory}. В работах \cite{Ground2010pmJ, Correlation2005SG} модель Эдвардса-Андерсона (ЭА) определяется как модель с бимодальным распределением обменных интегралов. Модуль обменного интеграла $J$ принимает одинаковые значения для всех взаимодействий между спинами в ближайшем окружении. В модели ЭА обменная константа $J$ может принимать только два возможных значения ''+1'' или ''-1''. 

В работах \cite{Feigelman1985Hierarchical, Murani1977Spin, Deryabin1983Features} ЭА-модель применяется для описания результатов экспериментов с $FeAu$, $FeNiMn$, $PtMnFe$ при отличных от нуля температурах.  Таких проявлений частично неупорядоченной или неупорядоченной конденсированной материи очень много в реальной природе, наверное даже много больше, чем теоретических моделей таких состояний \cite{ziman1979models}. В распоряжении теоретика имеется только небольшое число точно решаемых теоретических моделей \cite{baxter2016exactly}, которые разрабатывались для понимания структуры, различных свойств и в целом природы материи \cite{ziman1979models}. На исследование состояния спинового стекла было потрачено множество усилий \cite{binder1986spin, fischer1993spin, young1998spin, bezzub1983}. Модель спинового стекла с конечным радиусом взаимодействия оказалась сложной задачей для систем бесконечного числа частиц \cite{kor1989}. Некоторые теоретические вопросы до сих пор остаются открытыми для моделей бесконечного радиуса\cite{sherrington1975solvable}. Предпринимались попытки получить решение в модели конечного радиуса также и методами среднего поля \cite{belokon2006, belokon2001funk}. 

В данной работе методами Монте-Карло и исчерпывающего перечисления для конечного числа спинов Изинга в модели ограниченного радиуса (ближайших соседей) приближенно для образцов относительно большого числа частиц $N=40\times40$ и строго для малого числа частиц $N=8\times8$, получены критические температуры для заданного среднего значения обменной константы $P_+$. Точное решение позволило вычислить статистические суммы для систем конечного числа спинов в модели Изинга с отрытыми граничными условиями. Мы использовали результаты, приведенные в работе \cite{makarov2019} для численного расчета параметра фрустраций исследуемых моделей. Все это позволило построить теоретические фазовые магнитные диаграммы для ферромагнитной фазы, антиферромагнитной фазы и спинового стекла, в том числе и во внешнем магнитном поле.
