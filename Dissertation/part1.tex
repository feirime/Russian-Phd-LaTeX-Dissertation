\chapter{Модель}\label{ch:ch1}

\section{Модель Изинга на простой квадратной решетке}
Модель Изинга использовалась для исследования условий существования ферромагнетизма, антиферромагнетизма и спинового стекла. Эти термодинамические состояния могли быть обнаружены для исследуемых систем ниже найденных нами критических температур в зависимости от знака обменной константы взаимодействующих со своими ближайшими соседями $N$ спинов Изинга $S_i$, расположенных в узлах квадратной решетки. Ферромагнитные взаимодействия $J_{ij}=+1$ и антиферромагнитные взаимодействия $J_{ij}=-1$ случайным образом равномерно были распределены для всех $ij$-пар взаимодействующих спинов с заданной вероятность $P_+$. Энергия взаимодействия рассчитывалась по правилу
\begin{equation}
	E_t = -\sum_{<ij>}J_{ij} S_iS_j-HM_i,
	\label{eq:internal_energy}
\end{equation}
где $<ij>$ означает, что суммирование происходит только по ближайшим соседям, $H$ -- внешнее магнитное поле.  Спиновый избыток
\begin{equation}
	M_i = \sum_{j=1}^NS_j.
	\label{eq:spinex}
\end{equation}

Для каждого спинового избытка может существовать несколько различных значений энергии $E_t$, каждое со своим вырождением $g_t$. С учетом внешнего магнитного поля $H$ статсумма равна

\begin{equation}  
	Z=\sum_{l=1}^{2^N}\exp\left\{-\frac{E_l}{kT}\right\}=\sum_{t=1}^{\Omega}g_t \exp\left\{-\frac{E_t}{kT}\right\},
	\label{eq:partfunk}
\end{equation}
где $\Omega$ --- это общее число вырожденных состояний со всеми возможными значениями спинового избытка $M_i$, $k=1$ --- постоянная Больцмана. 