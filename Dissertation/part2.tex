\chapter{Методы расчёта}\label{ch:ch2}

\section{Методы расчёта критических температур для систем конечного числа спинов Изинга} Для строгого расчета теоретической фазовой диаграммы $T(P_+)$ были рассчитаны строго статистические суммы для десяти образцов со случайным равномерным распределением обменных констант по всем парам взаимодействующих спинов, а также выполняли Монте-Карло расчеты для такого же количества образцов. Необходимо отметить, что строгий расчет статистической суммы в модели Изинга для систем произвольного числа взаимодействующих спинов $N$ со случайным распределением обменных интегралов во внешнем магнитном поле представляет собой сложную задачу, для которой сегодня отсутствуют алгоритмы численного расчета, время работы которых полиномиально зависит от числа спинов $N$. Чтобы рассчитать статистическую сумму необходимо выполнить расчет энергии всех $2^N$ конфигураций в ходе исчерпывающего перечисления, т.е. существующие алгоритмы являются экспоненциально длительными.

Был разработан параллельный алгоритм (алг. \cref{lst:dos_exhaustive}), позволивший получить распределение Гиббса в модели конечного числа спинов (до $8\times8$)  с взаимодействием ближайших соседей на квадратной решетке со свободными граничными условиями. Метод исчерпывающего перечисления был использован в авторском параллельном суперкомпьютерном коде, с помощью которого проведены численные расчеты. Проводилась декомпозиция системы на элементы (одномерные цепочки). Плотное побитовое кодирование позволило хранить информацию (энергию, спиновый избыток, вырождение и др.) обо всех конфигурации подсистем из двух одномерных решёток спинов с учетом распределения обменных интегралов в подсистемах.  Энергия двух объединенных одномерных решеток в двумерную или присоединение одномерной решетки к двумерной будет следующей:
\begin{equation}
	E_{sum}  = E_{l}  + E_{r}  + E_{u},
	\label{eq:unification_energy}
\end{equation}

\noindent где $E_{l}$, $E_{r}$ -- энергии левой и правой решетки соответственно, $E_{u}$ энергия объединения решёток. 

Библиотека состояний подсистем позволяет значительно масштабировать число частиц. Мы использовали четное число спинов в системе $8\times8$, чтобы статистическая сумма содержала конфигурации с нулевым спиновым избытком для исключения влияния некомпенсированного спинового избытка на расчет термодинамических свойств во внешнем магнитном поле.

Основные положения метода Монте-Карло при расчете с помощью алгоритма Метрополиса, а также описание численного расчета термодинамических средних как для систем, где обменная энергия зависит от расстояния, так и для моделей ограниченного радиуса можно, например, найти, например, в серии наших работ  \cite{Shevchenko2017, makarov2019, Shevchenko2022, makarova2023}. Мы использовали следующие основные параметры алгоритма Метрополиса для расчета образцов $N=40\times40$ с заданным распределением обменных констант: количество шагов Монте-Карло составляло $10^5$ для приведения системы спинов в равновесное состояние; количество шагов для усреднения энергии и поиска кандидата на основное состояние выбиралось $10^6$; усреднение проводилось по всем полученным в ходе Монте-Карло семплирования конфигурациям, каждая из которых имела свое значение энергии.

\section{Алгоритм и особенности реализации}

Для строгого расчёта статистической суммы образов с заданным $P_+$ имитирующих ферромагнетик, антиферромагнетик и спиновое стекло был разработан параллельный высокопроизводительный алгоритм, который в последствии был реализован на языке CUDA (алг. \cref{lst:dos_exhaustive}).


\begin{ListingEnv}[!h]
	\captiondelim{ } % разделитель идентификатора с номером от наименования
	\caption{расчёт плотности состояний полным перебором}\label{lst:dos_exhaustive}
	\begin{Verb}
Конфигурация GPU:
Создание массива конфигураций 1D цепочки с использованием смещения битов
Создание G-тензора [конфигурация справа][E][M][множество простых чисел]
 FOR (Количество слоев в ячейке)
 {
  FOR (Каждая добавленная 1D цепочка)
  {
   FOR (Каждая конфигурация самого внешнего слоя нарастающей решетки)
   {
    FOR (Каждая конфигурация добавленной 1D цепочки)
     Вычисление энергии и магнитной восприимчивости
     Атомарное добавление вырождения нарастающей решетки в 
     G-тензор [конфигурация внешнего слоя][энергия][магнитная 
     восприимчивость][коэффициенты простых чисел]
    ENDFOR
   }
   ENDFOR
  }
  ENDFOR
 }
 ENDFOR
Расшифровка вырождений из множества простых чисел
Переформатирование данных из G-тензора в вырождение, энергию, плотность 
состояний магнитной восприимчивости
	\end{Verb}
\end{ListingEnv}

%\begin{algorithm}[H]
%	\textbf{INPUT:} Cell size, exchange integral distribution.\\
%	\textbf{OUTPUT:} Full density of states.
%	\begin{algorithmic}
%		\STATE {GPU configuration:}
%		\STATE {creating massive of configurations 1D chain by bit offset}
%		\STATE {creating G-tensor [config right][E][M][prime number set]}
%		\FOR {Number of layer in cell\\}
%		{
%			\FOR {Each added 1D chain\\}
%			{
%				\FOR {Each configuration of the outermost layer of the build-up lattice\\}
%				{
%					\FOR{Each configuration of added 1D chain\\}
%					\STATE {Calculate energy and magnetic susceptibility}
%					\STATE {Atomic add degeneration of the build-up lattice in G-tensor[configuration of outermost layer][energy][magnetic susceptibility][prime numer coefficients]}
%					\ENDFOR\\
%				}
%				\ENDFOR\\
%			}
%			\ENDFOR
%		}
%		\ENDFOR
%		\STATE{Deciphering degeneracies from the set of prime numbers}
%		\STATE{Reformating data from G-tensor to degeneracy, energy, magnetic susceptibility density of states}
%	\end{algorithmic}
%	\caption{Calculation density of states by exhaustive search.}
%	\label{algo:dos_exhaustive}
%\end{algorithm}

В расчетах на GPU узким местом является запись и чтение в global память, и, чтобы не тратить время на многократную запись накапливающихся в результате вычислений данных о вырождении, энергии и спиновом избытке  был создан массив G-tensor [config right][E][M][prime number set] с осями размера одномерной цепочки, максимальной энергии $\times 2 + 1$, максимального спинового избытка $\times 2 + 1$ и набор простых чисел для записи вырождения, соответственно. В этот тензор в процессе расчёта атомарно добавлялась информация о вырождении по координатам энергии $+ E_{max}$ и спинового избытка $+ M_{max}$.

Координата конфигурации цепочки была нужна для то чтобы знать какую конфигурацию правого слоя мы соединяем с новой 1D цепочкой справа и какие состояния всей левой решетки с этой конфигурацией связаны.

Поскольку вырождение состояний с одинаковой энергией и спиновым избытком могут принимать значения от $1$ до $\approx 2^{N}$, где $N$ -- размер решетки, то при размерах больше чем $8 \times 8$ вырождения уже не помещаются в целочисленный встроенный тип. Для решения этой проблемы получаемые при решении вырождения разбиваются на простые числа, а после всех расчётов расшифровываются с помощью китайской теоремы об остатках \cite{katz2007mathematics}.