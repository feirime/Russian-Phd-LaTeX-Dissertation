\chapter*{Заключение}                       % Заголовок
\addcontentsline{toc}{chapter}{Заключение}  % Добавляем его в оглавление

%% Согласно ГОСТ Р 7.0.11-2011:
%% 5.3.3 В заключении диссертации излагают итоги выполненного исследования, рекомендации, перспективы дальнейшей разработки темы.
%% 9.2.3 В заключении автореферата диссертации излагают итоги данного исследования, рекомендации и перспективы дальнейшей разработки темы.
%% Поэтому имеет смысл сделать эту часть общей и загрузить из одного файла в автореферат и в диссертацию:

%Основные результаты работы заключаются в следующем.
%\input{common/concl}
%И какая-нибудь заключающая фраза.

Мы разработали алгоритм строгого расчета статистической суммы для конечного числа спинов Изинга на простой квадратной решетке со свободными граничными условиями для систем с $N=8\times8$ спинов. Строгий численный расчет позволил получить информацию о спиновых избытках конфигураций для фиксированных концентраций обменных констант в модели Изинга. Это позволило рассчитать модуль намагниченности, который в ферромагнитной модели Изинга был отличен от нуля для конечных температур.

Мы использовали параметр фрустраций (\ref{eq:frustration_parameter}) для расчета критических температур перехода в антиферромагнитную и ферромагнитную фазы. Линейный рост параметра фрустраций в фазах ферромагнетика и антиферромагнетика обусловлен ростом числа возбуждений в основном и близких к основному низкоэнергетических состояниях. Переход в фазу спинового стекла в области точки перегиба приводит к выходу параметра фрустраций на насыщение. 

Исследование конфигураций основного состояния подтверждает, что параметр фрустраций выходит на насыщение при исчезновении перколяционного кластера. Дробление перколяционного кластера на мелкие кластеры увеличивает число возбуждений.

Переход в фазу спинового стекла из парамагнетика в отличном от нуля внешнем магнитном поле может быть определен по температуре максимума теплоемкости. Переход в фазу ферромагнетика из фазы спинового стекла происходит без видимых аномалий в температурном поведении теплоемкости при отличном от нуля внешнем поле (рис. \ref{fig:Stable_line}). Отсутствие аномалий в температурном поведении теплоемкости обусловлено тем, что фаза ферромагнетика является не спонтанной, а наведенной. Переход к наведенному ферромагнетизму контролируется энергией Зеемана. 

Проведенные численные эксперименты с помощью метода Метрополиса позволили получить температуры максимума теплоемкости в широком диапазоне значений обменных констант $P_+$, от антиферромагнитной области $P_+=0.0$, до ферромагнитной $P_+=1.0$. В области спинового стекла $P_+=0.5$ в отсутствии внешнего магнитного поля наблюдалось постоянство температур максимума теплоемкости и в точном решении, и в данных, полученных методом Монте-Карло, что может свидетельствовать об отсутствии эффекта количества частиц.

Понижение температуры максимума теплоемкости на $H-T$-диаграмме обусловлено увеличением термодинамической вероятности конфигураций с отличным от нуля спиновым избытком, т.е увеличением вклада энергии Зеемана в Гамильтониан.

Для ответа на вопрос о точном расчете критического значения обменной константы $P_+$ для фазовых переходов при $T\rightarrow0$ между состояниями ''$AFM-SG$'' и состояниями ''$FM-SG$'' необходимо исследование основных состояний. Однако, сегодня не существует каких либо подходов, схем или методов расчета, которые бы позволили разработать полиномиально-быстрые алгоритмы не только для поиска, но и для исследования статистических свойств основных состояний (энергии, вырождения, спинового избытка и других) даже в самой простой модели - модели Изинга произвольного распределения $J$. Сегодня отсутствует ответ на вопрос о возможном существовании способа расчета основного состояния в модели Изинга. Причина очевидна, основное состояние может быть получено только как точное решение. Любое приближение к основному состоянию позволяет получить только какое то из огромного множества возбужденных состояний. Важность исследования основных состояний обусловлена тем, что реализация основного состояния при $T\rightarrow0$ является достоверным событием.  Это состояние играет ключевую роль в термодинамике фазовых переходов.

Важные выводы в этой работе, на которых бы хотелось обратить внимание состоят в следующем. Первое --- линия нестабильности Алмейды-Таулеса не является следствием неравновесной физики или неэргодичности спинового стекла, поскольку выполнен строгий анализ всей статистической суммы, т.е. при бесконечном времени измерения.  Второе --- фазовая диаграмма была вычислена только с помощью компьютерных, вычислительных методов, алгоритмов исчерпывающего перечисления. Таким образом, компьютерные расчеты сегодня являются одним из важнейших инструментов теоретической физики для исследования существующих моделей, для которых строгое решение аналитическими методами пока не доступно.

Мы планируем исследования основных состояний, в том, числе во внешнем магнитном поле, т.е. с учетом энергии Зеемана, с помощью компьютерных методов, поскольку имеем доступ ко всем состояниям или конфигурациям.
